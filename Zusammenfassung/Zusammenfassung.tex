\documentclass{article}
\usepackage{a4wide}
\usepackage{titlesec}
\usepackage[ngerman]{babel}
\usepackage{hyperref}
\hypersetup{
	colorlinks,
	citecolor=black,
	filecolor=black,
	linkcolor=black,
	urlcolor=black
}
\title{Zusammenfassung: Praktische Philosophie}
\date{\today}
\author{ChatGPT}
\begin{document}
\shorthandoff{"}
\maketitle

\tableofcontents


\section{'Das Weltbild der Igel'}

\subsection{Zusammenfassung der "Die Igel am Fahrbahnrand"}

\begin{itemize}
	\item \textbf{Krise der Igel}: Die Geschichte handelt von einer Gruppe von fünf Igeln, die sich in einer Krise befinden, weil sie einer belebten Straße ausgesetzt sind. Die Igel sind nass, unruhig, belästigt von Parasiten und süchtig nach den Auswirkungen der Autoabgase.

	\item \textbf{Natur versus Anthropozän}: Es gibt ein brennendes Spannungsfeld zwischen der Wildheit der Igel und der von Menschen geprägten Welt der Schnellstraße. Die Igel versuchen, ihre instinktbasierte Strategie des Überquerens dieser ungewohnten, gefährlichen Umgebung zu verstehen und anzuwenden.

	\item \textbf{Folgen des Anthropozäns}: Die Konsequenzen ihrer Konfrontation mit der Straße sind tödlich. Zwei der Igel werden sofort überfahren. Ein anderer überschreitet die Straße, während die verbleibenden zwei kopflos zur Ausgangsposition zurückkehren.

	\item \textbf{Fremdartige Umwelt und Kampf ums Überleben}: Die Erzählung betont den widerspenstigen Überlebenskampf der Igel in einer Umgebung, die weit entfernt von ihrer natürlichen Landschaft ist. Sie schlafen neben Mülltonnen, essen unbekanntes Futter und sind mit der Verschmutzung und dem Krach der Straße konfrontiert.
\end{itemize}

\subsection{Mitleid und Moral: Eine Untersuchung der philosophischen Ethik}

\begin{itemize}
	\item \textbf{Kant und Mitleid}: Nach Kant ist Mitleid nur ein Gefühl, keine vernünftige Handlung. Im Gegensatz zu Kant argumentiert Schiller, dass Mitgefühl auch ein Ausdruck moralischen Handelns sein kann. Kant betont zwar die Erziehung des Gefühlslebens, sieht Tiere allerdings nur als moralisch unwesentlich und ihnen nur einen materiellen "Preis", keine Würde zu.

	\item \textbf{Schopenhauers Kritik an Kant}: Schopenhauer kritisiert Kants Ansicht, dass Tiere nicht das Objekt von Mitleid sein sollten, da sie kein Vernunftvermögen besitzen. Er hinterfragt Kants Bemühungen um Tierschutz als pädagogisch motiviert, indem er ironisch vorschlägt, dass Mitleid mit Tieren nur geübt wird.

	\item \textbf{Utilitarismus und Tiere}: Im Gegensatz zum kantischen Ansatz kalkuliert der Utilitarismus das Vergnügen und Leid aller Lebewesen, einschließlich der Tiere. Hier zeigt sich eine klare Diskrepanz zwischen der Philosophie im deutschsprachigen und englischsprachigen Raum. Berühmte Utilitaristen wie Bentham vergleichen die Missachtung von Tieren mit Rassismus und prägen Begriffe wie "Speziesismus".

	\item \textbf{Vertragstheorie und Tiere}: Diese Theorie argumentiert, dass moralische Pflichten auf Verträgen basieren, und da Tiere keine Verträge abschließen können, sind sie von moralischen Pflichten ausgeschlossen. Dies steht im direkten Widerspruch zur utilitaristischen Sichtweise, die die Fähigkeit zu Empfinden als ausschlaggebend für moralische Berücksichtigung sieht.
\end{itemize}

\subsection{Zusammenfassung von "Der Pluralismus von Moral und Vernunft"}

\begin{itemize}
	\item \textbf{Einführung}: Der Text argumentiert, dass die moralischen Theorien oft eine eingeschränkte Sicht auf Tiere und das Leiden haben, im Gegensatz zu unseren alltäglichen moralischen Einstellungen. Es wird darauf hingewiesen, dass das philosophische Nachdenken über die Moral komplex und nicht auf einen einzigen Aspekt reduziert sein sollte.

	\item \textbf{Gefühl und Verstand}: Hier wird die Idee diskutiert, dass die allgemeine Vernachlässigung von Gefühlen in der Philosophie mit dem geringen Ansehen von Tieren zusammenhängen könnte. Es wird argumentiert, dass Verstand und Gefühl unterschiedliche, aber wichtige Aspekte der Vernunft sind und beide zur Erkenntnis beitragen können.

	\item \textbf{Vernunft}: Diese Diskussion hebt die Wichtigkeit eines ganzheitlichen, ausbalancierten Verständnisses von Vernunft hervor, das sowohl Verstand als auch Emotionen einbezieht. Vernunft sollte nicht als rein verstandbasiert verstanden werden, sondern als ein Zusammenspiel von Verstand und Gefühl, wobei beide zur Erkenntnis beitragen. Anerkennung dieser Dualität kann dazu führen, dass Gefühle nicht nur als pädagogischer Anreiz, sondern als Teil unseres Vernunftvermögens verstanden werden können.
\end{itemize}

\subsection{Zusammenfassung: Literatur und Gefühl}

\begin{itemize}

	\item \textbf{Literatur und Emotionsaktivierung}: Der Text enthält eine Diskussion über die Fähigkeit der Literatur, unsere Gefühle zu aktivieren und warum das wichtig ist. Insbesondere wird angenommen, dass die literarische Darstellung konkreter Einzelschicksale, wie in dem erwähnten Beispiel mit den Igeln, stärkere emotionale Reaktionen hervorruft als abstrakte Statistiken oder Daten. In anderen Worten erlaubt uns die Literatur, Empathie und Moral zu kultivieren und dadurch eine tiefere Erkenntnis der Welt zu gewinnen.

	\item \textbf{Arten von Gefühlen}: Die Passage untersucht die Nuancen und die Vielzahl der in der literarischen Passage beschriebenen Emotionen der Igel. Die Diskussion unterscheidet zwischen verschiedenen Arten von Gefühlen: Leibliche Empfindungen, Gesamtleibesempfindungen, Stimmungen und Emotionen. Es wird betont, dass diese Gefühle, insbesondere Emotionen, einen inhaltlichen Bezug zur Welt haben.

	\item \textbf{Ausdruck von Gefühlen in der Literatur}: Die Passage diskutiert, wie die Literatur Gefühle ausdrückt und "performt", insbesondere durch die Verwendung literarischer Stilmittel wie Assonanzen, Alliterationen, Wiederholungen, Ellipsen und andere.

	\item \textbf{Emotionstheorien}: In der Passage werden verschiedene Theorien der Emotionen diskutiert. Insbesondere wird die narrative Theorie betont, die die temporale und thematische Entwicklung von Emotionen hervorhebt. Aber auch andere Theorien, wie die Körpertheorie, Empfindungstheorie, Verhaltenstheorie und kognitive Theorie, werden erwähnt. Emotionen werden als inhaltliche Stellungnahmen zur Welt, als Ausdruck unserer Werte und Persönlichkeit und als "Alarmmechanismen" gesehen.

\end{itemize}

\subsection{Empathie oder Sympathie?}

\begin{itemize}
	\item \textbf{Arten von Fremdwahrnehmung}: Die Phänomenologie unterscheidet zwischen verschiedenen Formen der Fremdwahrnehmung, darunter Empathie (Nachfühlen), Sympathie (Mitfühlen), reine Wahrnehmung, Ansteckung und Einsfühlung. Gegenwärtig tendiert man dazu, diese Unterschiede zu ignorieren und alles als Empathie zu betrachten.

	\item \textbf{Empathie}: Empathie bezieht sich auf die Fähigkeit, sich in die Gefühle des Anderen hineinzuversetzen und vorzustellen, wie es ihm oder ihr dabei geht. Dabei gibt es drei Stufen: die Identifikation des Gefühls des Anderen, eine sinnliche Ausmalung dieses Gefühls und ein Stück weit Distanzierung davon. Dies erlaubt ein Bewusstsein des Unterschieds zwischen Ich und Du. Empathie ist nicht grundsätzlich moralisch gut, denn auch negative Gefühle wie Sadismus können auf Empathie aufbauen.

	\item \textbf{Sympathie}: Im Gegensatz zur Antipathie ist Sympathie ein gleichgerichtetes Gefühl; man empfindet Leiden mit und Freude freudig. Sie beinhaltet eine negative Stellungnahme zu Leiden und eine Mitvollziehung dieser Stellungnahme. Damit hat das sympathische Gefühl eine moralische Qualität, obwohl Moral mehr ist als bloßes Mitleid. Dabei geht es nicht nur um Empathie, sondern auch um das Nachfühlen von Leiden und Freuden des Anderen.
\end{itemize}

\subsection{Aus der Sicht der Igel von Peter Kurzeck}

\begin{itemize}
	\item \textbf{Empathie und Sympathie für Tiere}: Im Text wird das Verständnis und Mitgefühl für Tiere, insbesondere Igel hervorgehoben. Der Autor wirft Licht auf das durch Menschen verursachte Leiden dieser Tiere, das oft übersehen wird.
	\item \textbf{Anthropozentrismus und Dezentrierung}: Der Text stellt eine erdgeschichtliche Erzählung dar, die das menschliche Weltbild herausfordert und verfremdet. Er betont den ersten Blickwinkel der Igel und zwingt die Leser, eine Sichtweise zu betrachten, die sich von der üblichen menschenzentrierten Perspektive unterscheidet.
	\item \textbf{Kritik der menschlichen Aktivitäten aus der Perspektive der Tiere}: Das Narrativ stellt dar, wie Igel menschliches Verhalten, vor allem ihre Zunahme und Ausbreitung, kritisch beobachten und beurteilen. Negativ formulierte Werturteile der Igel geben die verschlechternde Einstellung gegenüber Menschen wieder.
	\item \textbf{Identifikation mit Tieren}: Kurzeck zeigt nicht nur Empathie (Fähigkeit, Gefühle und Empfindungen anderer zu verstehen), sondern auch Sympathie (Mitgefühl, Unterstützung). Durch die Übernahme der negativen Urteile der Igel zeigt er, dass er sich mit ihnen identifizieren kann.
\end{itemize}

\end{document}
