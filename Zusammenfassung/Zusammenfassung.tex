\documentclass{article}
\usepackage{a4wide}
\usepackage{titlesec}
\usepackage[ngerman]{babel}
\usepackage{hyperref}
\hypersetup{
	colorlinks,
	citecolor=black,
	filecolor=black,
	linkcolor=black,
	urlcolor=black
}
\title{Zusammenfassung: Praktische Philosophie}
\date{\today}
\author{ChatGPT}
\begin{document}
\shorthandoff{"}
\maketitle
\tableofcontents
\newpage



\section{Einleitung: Grundkurs Ethik}

\begin{itemize}
	\item \textbf{Ethik:} Betrachtet als Teilgebiet der Praktischen Philosophie, untersucht Ethik die normativen Grundlagen menschlichen Handelns. Sie befasst sich mit der zentralen Frage "Was soll ich tun?" und zielt darauf ab, Handlungsorientierungen und moralische Bewertungen des Verhaltens zu begründen.

	\item \textbf{Theoretische vs. Praktische Urteile:} Theoretische Urteile beschreiben die Welt, wie sie ist ("Es regnet."), während praktische Urteile normative Aussagen über unser Handeln treffen ("Es ist gut, den Mantel anzuziehen."). Beide Arten von Urteilen sind jedoch eng miteinander verbunden, da theoretische Urteile oft Handlungen leiten und selbst Ergebnisse normativ geleiteter Forschung sind.

	\item \textbf{Zweckrationalität vs. Praktische Vernunft:} Unterscheidet zwischen Handlungen, die lediglich Mittel zu einem anderen Zweck sind (zweckrational), und solchen, die einen intrinsischen Wert haben und um ihrer selbst willen erstrebenswert sind (praktisch vernünftig).

	\item \textbf{Ethik des guten Lebens:} Diese Sparte der Ethik befasst sich mit allgemeinen Einsichten in das gute menschliche Leben, das über ein teleologisches Verständnis von Handlungszwecken hinausgeht und den Wert letzter Zwecke betont, die nicht in einem infiniten Regress münden.

	\item \textbf{Moralphilosophie:} Die Disziplin, die den gebotenen Respekt für andere thematisiert, und dabei drei Hauptpositionen diskutiert: Utilitarismus (Maximierung des Gesamtnutzens), Kantianismus (Respekt für die Autonomie eines jeden) und Aristotelismus (Rücksicht auf das gute Leben eines jeden).

	\item \textbf{Politische Philosophie:} Untersucht die letzten moralischen und eudämonistischen Zwecke gesellschaftlicher Institutionen und konzentriert sich auf Konzepte von Gerechtigkeit und Gemeinwohl.

	\item \textbf{Angewandte Ethik:} Wendet ethische Theorien auf konkrete gesellschaftliche Kontroversen an und überprüft ihre Anwendbarkeit und Aussagekraft in realen Problemstellungen.
\end{itemize}
\newpage

\section{Ein Gespräch mit Peter Kurzeck}

\begin{itemize}
	\item \textbf{Resonanzthese:}
	      Die Resonanzthese besagt, dass Menschen die emotionale Interaktion mit der Natur benötigen, um vollständig Mensch zu sein. Es geht um das Mitschwingen mit natürlichen Rhythmen und das Aufgehen in der Stimmung einer Landschaft, wodurch sich Menschen mit der Erde verbunden fühlen und ohne welche sie nicht gedeihen können.

	\item \textbf{Personifikation der Natur:}
	      Kurzeck nutzt die Personifikation der Natur als stilistisches Mittel, um die Verbindung zwischen Mensch und Natur zu verstärken und zu verdeutlichen. Die Natur wird als Du konstruiert, das mit den Menschen sprechen und auf das sie hören sollten. Dieses Stilmittel soll eine tiefere emotionale Reaktion beim Leser hervorrufen und den Wunsch nach Erhalt der Natur stärken.

	\item \textbf{Vorher-Nachher-Kippbilder-Technik:}
	      Diese Technik verwendet kontrastive Bilder der Natur, um Gefühle des Verlusts zu intensivieren. Sie kontrastiert auf schmerzvolle Weise die Natur in ihrer früheren Vielfalt und Schönheit mit ihrem aktuellen, oft degradierten Zustand, um das Bewusstsein und die Wertschätzung für das zu schärfen, was verloren gegangen ist.

	\item \textbf{Engagierte Literatur:}
	      Kurzecks Werk wird als engagierte Literatur betrachtet, das heißt, sein Ziel ist es, durch Literatur die Welt zu verbessern, indem er wesentliche Anliegen aufgreift und diskutiert. Seine engagierte Literatur versucht, durch das Wecken von Empathie und Bewusstsein für ökologische und soziale Themen einen Wandel anzuregen.

	\item \textbf{Verhältnis Literatur und Philosophie:}
	      Kurzecks Literatur lässt sich in einer philosophischen Perspektive verstehen, denn sie bildet ihre These nicht durch direkte Behauptungen, sondern durch plastisches Zeigen mit Erfahrung und Gefühl ab. Trotz des emotionalen Charakters trägt es zu einem tieferen Verständnis und zur Diskussion philosophischer Fragen der Ethik und des menschlichen Zustandes bei.
\end{itemize}

\subsection{Bezug zur Naturethik}
Der Text verknüpft direkt mit der Naturethik, indem er den intrinsischen Wert der Natur betont und die Wechselwirkungen zwischen Menschen und ihrer natürlichen Umgebung hervorhebt. Durch die starke emotionale Ladung und die Personifikation der Natur regt die Literatur zu einem ethischeren Umgang mit der Umwelt an und unterstützt die Idee, dass die Natur Rechte haben könnte oder als Teil eines moralischen Gesprächs betrachtet werden sollte.

\newpage
\section{Das Weltbild der Igel}

\begin{itemize}

	\item Mitleid und Moral
	      \begin{itemize}
		      \item Kants Sicht: Mitleid ist lediglich ein Gefühl, das nicht aus Vernunft stammt.
		      \item Schillers Kritik: Er betont die Bedeutung des Mitleids und seine moralische Relevanz.
		      \item Kants Erziehungsidee: Die Kultivierung unserer Gefühle soll moralisches Handeln erleichtern.
		      \item Kants Tierethik: Ablehnung von Tierquälerei aus pädagogischer Perspektive.
		      \item Schopenhauers Kritik: Bezweifelt Kants Ansicht, dass Mitleid mit Tieren nur Übungszwecken dient.
		      \item Kognitivistisch vs. Nicht-Kognitivistisch: Debattiert über die Quellen moralischen Handelns, sei es Vernunft oder Emotion.
	      \end{itemize}
	\item Utilitarismus versus Vertragstheorie
	      \begin{itemize}
		      \item Utilitarismus: Verfolgt eine Erhöhung der Lust-über-Unlust-Bilanz unter Berücksichtigung leidensfähiger Tiere.
		      \item Vertragstheorie: Nur Wesen, die Verträge schließen können, haben moralischen Anspruch, was Tiere ausschließt.
	      \end{itemize}

	\item Moral- und Vernunftpluralismus
	      \begin{itemize}
		      \item Pluralistische moralische Kultur: Leiden ist unabhängig vom Leidenden negativ.
		      \item Notwendigkeit, die moralische Komplexität im Alltag in der Philosophie widerzuspiegeln und nicht auf einzelne Aspekte zu reduzieren.
	      \end{itemize}

	\item Emotionstheorien
	      \begin{itemize}
		      \item Mehrkomponententheorie: Emotionen bestehen aus verschiedenen Komponenten wie körperlichen Zuständen und inhaltlichen Stellungnahmen.
		      \item Narrative Theorie: Sieht Emotionen als zeitlich entwickelnde Geschichten, die inhaltliche Stellungnahme und moralische Qualität enthalten.
		      \item Kognitive Theorie: Betont, dass der inhaltliche Bezug von Emotionen sie auszeichnet und sie unsere Werte und Persönlichkeit offenbaren.
	      \end{itemize}

	\item Empathie und Sympathie
	      \begin{itemize}
		      \item Empathie: Sich anschaulich in das Gefühlsleben Anderer hineinversetzen unter Bewahrung des Ich-Du-Bewusstseins.
		      \item Sympathie: Ein Mitfühlen, das moralisch wertend ist und Leiden oder Freude teilt.
		      \item Rolle der Literatur: Stimuliert durch Empathie und kultiviert durch Sympathie unsere moralischen Gefühle.
	      \end{itemize}

\end{itemize}

\subsection*{Gegenargumente und Probleme}
\begin{itemize}
	\item Anthropozentrische Ethik: Vernachlässigung der moralischen Relevanz nichtmenschlichen Lebens.
	\item Praktische Philosophie und Tierethik: Konzeptuelle Vernachlässigung des Leids nichtmenschlicher Wesen in moralischen Überlegungen.
	\item Reduktionistische Sicht auf Emotionen: Emotionen werden oft auf einen Aspekt wie körperliche Erregung oder inhaltliche Stellungnahme reduziert, was ihre komplexe Natur ignorieren kann.
	\item Überbetonung des Empathiebegriffs: Vernachlässigung der differenzierten Betrachtung anderer Formen der Fremdwahrnehmung.
\end{itemize}

\subsection*{Bezug zur Naturethik}
\begin{itemize}
	\item Die Darstellung von Tierleben und Leiden in der Literatur kann ein ethisches Bewusstsein für die Belange der Natur fördern.
	\item Die Fähigkeit der Literatur, Empathie und Sympathie zu wecken, trägt dazu bei, die anthropozentrische Sicht zu hinterfragen und zu einem pluralistischen Verständnis von Moral und Vernunft beizutragen.
	\item Durch das Nach- und Mitfühlen von Tieren in literarischen Werken wird eine Tiefe der moralischen Reflektion erreicht, die rein rationale Ansätze oft missen.
\end{itemize}

\newpage
\section{Erkenntnis in Wissenschaft, Philosophie und Dichtung}

\begin{itemize}
	\item \textbf{Personifikation der Natur}: Bezieht sich auf die metaphorische Zuschreibung menschlicher Eigenschaften an nicht-menschliche Wesen oder Dinge, was eine Fehlzuschreibung im wörtlichen Sinne darstellt, aber im übertragenen Sinne wichtige Erkenntnisse und emotionale Bezüge schaffen kann.

	\item \textbf{Spielarten der Personifikation}: Verschiedene Arten, in denen die Personifikation in der Sprache Anwendung findet, wie etwa die Projektion subjektiver Befindlichkeiten oder symbolische Darstellungen.

	\item \textbf{Evokation und Diskurs}: Unterscheidung zwischen metaphorischer bzw. evokativer Rede, die nicht-propositionales Wissen vermittelt (etwa ein Gefühl oder eine Erfahrung), und diskursiver Rede, die propositionales Wissen (Aussagesätze und Fakten) vermittelt.

	\item \textbf{Propositionale und nicht-propositionale Erkenntnis}: Es wird zwischen aussageförmigem Wissen ("Wissen dass") und demonstrativem, erlebnisorientiertem Wissen ("Wissen wie") unterschieden, wobei letzteres nicht immer verbal ausgedrückt wird und oft in der Kunst und Literatur zu finden ist.

	\item \textbf{Komplementarität der Erkenntnisarten}: Sowohl propositionales als auch nicht-propositionales Wissen sind unverzichtbar und ergänzen einander, um ein ganzheitliches Verständnis von Welt und Mensch zu erreichen.

	\item \textbf{Physiozentrik vs. Anthropozentrik}: In der Naturethik wird diskutiert, ob die Natur einen Eigenwert hat, der unabhängig vom Nutzen für den Menschen ist (Physiozentrik), oder ob ihr Wert sich ausschließlich über ihren Nutzen für den Menschen bestimmt (Anthropozentrik).

	\item \textbf{Eudaimonistischer Eigenwert der Natur}: Eine Position, die der Natur einen Wert zuweist, der zwar relational und nicht moralisch intrinsisch ist, aber über einen instrumentellen Wert hinausgeht und sich auf das Wohlergehen und ein gutes Leben des Menschen bezieht.

	\item \textbf{Teleologisches Argument}: Argument, das der Natur Zweckhaftigkeit zuschreibt und auf Basis dessen eine moralische Schutzwürdigkeit der natürlichen Zwecke fordert.

	\item \textbf{Holistisches Argument}: Argument, das die Einheit von Mensch und Natur betont und aus dieser Verbindung eine moralische Verpflichtung zum Naturerhalt ableitet.
\end{itemize}

\subsection*{Gegenargumente und Probleme}
\begin{itemize}
	\item \textbf{Teleologisches Argument}: Problematisch ist die Vermischung von funktionalen und praktischen Zwecken. Funktionale „Zwecke“ sind nicht mit moralischer Schutzwürdigkeit gleichzusetzen, da sie im System der Natur keine subjektive Zweckhaftigkeit aufweisen wie menschliches Handeln.

	\item \textbf{Holistisches Argument}: Hierbei wird die Vieldeutigkeit des Satzes „Der Mensch ist Teil der Natur“ vernachlässigt. Der rein metaphorische Bezug zum Miteinander in der Interaktion mit der Natur wird mit einem moralischen oder ontologischen Status vermengt. Dies könnte zu problematischen ethischen Konsequenzen führen.

	\item \textbf{Problem der Personifikation}: Die literarische Personifikation der Natur kann zu Missverständnissen führen, wenn sie als wörtliche Wahrheit statt als metaphorischer Ausdruck der menschlichen Erfahrung interpretiert wird.
\end{itemize}

\subsection*{Bezug zur Naturethik}
Die Naturethik beschäftigt sich mit der Frage nach dem moralischen Eigenwert der Natur. Die Diskussion um Personifikation und Metaphorik ist bedeutsam, da sie unsere Sichtweise auf die Natur und damit unser ethisches Handeln ihr gegenüber beeinflusst. Die Personifikation der Natur kann aufzeigen, dass der Natur auch außerhalb rein instrumenteller Nutzung Wert beigemessen wird, was für ein breiteres, eudaimonistisches Verständnis der Mensch-Natur-Beziehung plädiert.


\newpage
\section{Die unersetzbare Schönheit der Natur}

\begin{itemize}
	\item Ästhetische Resonanz mit der Natur
	      \begin{itemize}
		      \item Beschreibt ein intensives, emotionales und zugleich spontanes Echo innerhalb des Menschen auf die Schönheit der Natur.
		      \item Sie ist nicht bloß subjektiv, sondern enthält einen tieferen Verweis auf die Verbundenheit des Menschen mit der Natur, die die Person auch physisch und psychisch miterleben kann.
	      \end{itemize}

	\item Immersive Qualität von Landschaften
	      \begin{itemize}
		      \item Landschaften sind aufgrund ihrer Räumlichkeit, in der Menschen physisch existieren und interagieren, einzigartig in ihrer Fähigkeit, Individuen vollständig einzubeziehen.
		      \item Diese Qualität ermöglicht es Individuen, eine tiefgreifende Verbundenheit mit der Umwelt zu erleben, welche sich in Form von Stimmungen manifestiert.
	      \end{itemize}

	\item Symbolischer Reichtum der Natur
	      \begin{itemize}
		      \item Naturphänomene können tiefe symbolische Bedeutungen für Menschen bereithalten, wie z.B. der Wechsel der Jahreszeiten, der mit menschlichen Lebenszyklen korrespondiert.
		      \item Diese symbolischen Aspekte bereichern die menschliche Erfahrung und tragen dazu bei, dass die Natur um ihrer selbst willen geschätzt wird.
	      \end{itemize}

	\item Notwendigkeit der Personifikation der Natur
	      \begin{itemize}
		      \item Besagt, dass durch metaphorische Personifikation der Natur ein Zugang zu tiefen ästhetischen Erfahrungen ermöglicht wird.
		      \item Diese Verfahrensweise wird als unerlässlich für die vollständige menschliche Erfahrung betrachtet und trägt substantiell zu einem reichen menschlichen Leben bei.
	      \end{itemize}

	\item Kritik am künstlichen Ersatz der Natur
	      \begin{itemize}
		      \item Verweist auf die Probleme eines möglichen Ersatzes echter Natur durch künstliche Strukturen, wie zum Beispiel plastische Nachbildungen oder Virtual-Reality-Umgebungen.
		      \item Solche Imitationen könnten nie das ursprüngliche Erlebnis echter Natur ersetzen, insbesondere da der Mensch sich der Unechtheit bewusst wäre.
	      \end{itemize}
\end{itemize}

\subsection*{Gegenargumente und Probleme}
\begin{itemize}
	\item Einwände könnten die subtile Unterscheidung zwischen realer, naturbelassener Umwelt und einer künstlich geschaffenen Atmosphäre in Frage stellen.
	\item Künstliche Umgebungen könnten in technologischer Hinsicht weiterentwickelt werden, um die Erlebnisqualität zu verbessern und die Kluft zur echten Natur zu verringern.
	\item Ethische Debatten könnten argumentieren, dass die zur Schau gestellte Naturschönheit privilegiert und nicht allen Menschen zugänglich ist, und somit gesellschaftliche Ungleichheiten verstärkt.
\end{itemize}

\subsection*{Bezug zur Naturethik}
In Bezug auf die Naturethik unterstreicht der Text die moralische Verpflichtung, natürliche Schönheit zu erhalten und weiter zu fördern, da diese für das menschliche Wohlbefinden und die Qualität des menschlichen Lebens wesentlich ist. Eine ästhetische Wertschätzung der Natur kann zu einem umweltbewussteren Verhalten und zu einem Respekt vor der natürlichen Welt führen, was zentrale Anliegen der Naturethik sind.


\newpage
\section{Utilitarismus klassisch \& modern: John Stuart Mill \& Peter Singer}

\subsection{Wichtige Konzepte und ihre Erklärungen}
\begin{itemize}
	\item Konsequentialismus: Eine Handlung wird als moralisch richtig oder falsch basierend auf ihren Konsequenzen bewertet, nicht auf Basis von Absichten oder Charakter.

	\item Maximierungsprinzip: Die moralisch richtige Handlung ist diejenige, die den besten Gesamtnutzen oder das größte Gesamtglück produziert.

	\item Nützlichkeitsprinzip: Der Wert einer Handlung wird nach ihrem Beitrag zum Wohlergehen oder Glück aller Betroffenen bemessen.

	\item Aggregationsprinzip: Das Gesamtglück wird als Summe des individuellen Nutzens aller von der Handlung betroffenen Individuen verstanden.

	\item Hedonistischer/Präferenzutilitarismus: Beim hedonistischen Utilitarismus steht die Maximierung von Lust und die Minimierung von Schmerz im Vordergrund, während beim Präferenzutilitarismus Präferenzen und Interessen auch unabhängig von Lustgefühlen einbezogen werden.

	\item Unparteilichkeitsprinzip: Bei der Bewertung von Handlungen sollen alle Betroffenen gleich behandelt werden; persönliche Bindungen oder Präferenzen spielen keine Rolle.

	\item Speziesismus: Die Diskriminierung von Wesen aufgrund ihrer Spezieszugehörigkeit ist moralisch willkürlich und daher abzulehnen.

	\item Aktive vs. Passive Euthanasie: Aktive Euthanasie bedeutet das direkte Herbeiführen des Todes, während passive Euthanasie das Sterbenlassen ohne lebensverlängernde Maßnahmen bedeutet.

	\item Ersetzbarkeitsthese: Nach der utilitaristischen Totalansicht können Individuen unter bestimmten Bedingungen ersetzt werden, wenn dies die allgemeine Glückssumme erhöht.
\end{itemize}

\subsection{Probleme und Gegenargumente}
\begin{itemize}
	\item Gleichheit verschuldet: Der Utilitarismus scheint unvereinbar mit der Idee unveräußerlicher gleicher Rechte zu sein, da er auf einer Kosten-Nutzen-Analyse basiert und individuelle Rechte zugunsten des Gesamtnutzens übergehen kann.

	\item Entfremdung und Integrität: Die utilitaristische Forderung nach einer unparteilichen Sicht kann zu einer Entfremdung von persönlichen Bindungen und einer Gefährdung der eigenen Integrität führen.

	\item Keine Handlungs- und Verantwortungstheorie: Utilitarismus konzentriert sich auf die Folgen von Handlungen und vernachlässigt die Motive und Verantwortlichkeiten, die für moralisches Handeln entscheidend sind.

	\item Fehlbarkeit der moralischen Arithmetik: Die Bewertung von Handlungen kann aufgrund unsicherer oder unvorhersehbarer Folgen fehleranfällig sein, was die zuverlässige Anwendung des Utilitarismus untergräbt.
\end{itemize}

\subsection{Bezug zur Naturethik}
In der Naturethik wird der utilitaristische Ansatz dahingehend kritisiert, dass er die empfindungsfähige wie auch die nicht-empfindungsfähige Natur in erster Linie als Mittel zur Nutzensteigerung sieht und nicht einen inhärenten Wert in ihr erkennt. Laut Mill kann z.B. die ästhetische Erfahrung von Natur zur Steigerung des menschlichen Glücks beitragen und so indirekt im utilitaristischen Kalkül berücksichtigt werden; jedoch fehlt eine Betrachtung, die der Natur einen Wert jenseits ihrer Nützlichkeit für den Menschen zuschreibt.


\newpage
\section{Kantianismus klassisch}

\begin{itemize}
	\item \textbf{Guter Wille:} Im Zentrum von Kants Ethik steht die Idee, dass allein der gute Wille ohne Einschränkung als gut zu betrachten ist. Andere Qualitäten wie Talente oder Glücksgaben können nur dann als gut bewertet werden, wenn sie mit einem guten Willen einhergehen. Der gute Wille ist somit das höchste Gut und unabhängig von den Handlungsfolgen.

	\item \textbf{Pflicht versus Neigung:} Moralisches Handeln leitet sich für Kant aus der Pflicht und nicht aus Neigung oder Selbsterhaltungstrieben ab. Während Neigungen bedingt und veränderlich sind, ist Pflicht die Notwendigkeit, aus Achtung vor dem moralischen Gesetz zu handeln. Der innere Wert einer Handlung liegt in der Absicht, aus Pflicht zu handeln, und nicht in ihren Konsequenzen.

	\item \textbf{Kategorischer Imperativ:} Der kategorische Imperativ ist das grundlegende Prinzip der Kantischen Moralphilosophie. Es verlangt, dass eine Maxime (Handlungsregel) so gewählt wird, dass sie als allgemeingültiges Gesetz gelten könnte. Dies prüft, ob eine Handlung objektiv und universal als moralisch gelten kann.

	\item \textbf{Autonomie und Selbstgesetzgebung:} Autonomie bezieht sich auf die Fähigkeit, sich selbst Gesetze zu geben, die nicht durch äußere Einflüsse bestimmt sind. Der Mensch als vernunftbegabtes Wesen muss frei von äußeren Zwängen und in Einklang mit dem kategorischen Imperativ entscheiden können.

	\item \textbf{Menschheit als Zweck an sich:} Jeder Mensch muss immer zugleich als Zweck und nie bloß als Mittel zum Zweck behandelt werden. Dies stellt die unantastbare Würde des Menschen sicher und verbietet zum Beispiel die Instrumentalisierung von Personen.

	\item \textbf{Reich der Zwecke:} In der Kantischen Ethik ist das "Reich der Zwecke" eine ideale Gemeinschaft rationaler Wesen, in der jedes Individuum seine Handlungen nach Maximen wählt, die zugleich für alle gelten könnten.
\end{itemize}

\subsection{Gegenargumente und Probleme}
\begin{itemize}
	\item \textbf{Rigorismus:} Kritiker werfen Kant vor, seine Ethik sei zu starr und lasse keine Ausnahme zu ungünstigen Handlungsfolgen zu, selbst wenn diese schwerwiegend sind. Dadurch könnten moralisch fragwürdige Situationen entstehen.

	\item \textbf{Formalismus:} Der kategorische Imperativ wird als möglicherweise zu abstrakt und formal kritisiert, was die Ableitung konkreter Handlungsanweisungen erschwert.

	\item \textbf{Gesinnungsethik:} Kant konzentriert sich auf die Absicht hinter einer Handlung statt auf die Folgen, was zu einer Vernachlässigung der Verantwortung für die Konsequenzen von Handlungen führen kann.

	\item \textbf{Pflichtethik:} Die rigide Fokussierung auf Pflicht kann menschliche Emotionen und Begehren übergehen, die im Ethos anderer philosophischer Strömungen wie dem Utilitarismus oder dem Aristotelismus größeren Stellenwert haben.

	\item \textbf{Autonomielastigkeit:} Kants Ethik überbetont Autonomie gegenüber anderen Werten und beschränkt die Moral auf den Umgang zwischen autonomen Wesen, was Implikationen für Fragen der Tier- und Umweltethik hat.
\end{itemize}

\subsection{Bezug zur Natur- und Tierethik}
Kants Ethik sieht Tiere und die nicht-menschliche Natur nicht als moralische Zwecke an sich, sondern erkennt ihnen ausschließlich instrumentellen Wert zu. Tiere und Natur haben einen Wert nur insoweit, als sie für die kulturelle und moralische Entwicklung des Menschen bedeutsam sind. Ethische Pflichten bestehen demnach primär gegenüber vernunftbegabten Wesen.


\newpage
\section{Kantianismus modern: John Rawls}

\subsection{Hauptkonzepte der Theorie}

\begin{itemize}
	\item \textbf{Gerechtigkeit als erste Tugend sozialer Institutionen:} Rawls vertritt die Auffassung, dass Gerechtigkeit die fundamentale Tugend der Grundstruktur einer Gesellschaft ist, welche die Freiheit und Gleichheit der Bürger in einer kooperativen Gemeinschaft gewährleisten soll.
	\item \textbf{Überlegungsgleichgewicht (Reflective Equilibrium):} Dies ist eine Methode zur Abstimmung von moralischen Theorien und Prinzipien mit unseren moralischen Intuitionen. Bei Konflikten zwischen Theorie und Intuition sollten wir die besser begründbare Position wählen und entsprechende Anpassungen vornehmen, um ein kohärentes Gerechtigkeitsbild zu erreichen.
	\item \textbf{Idealer Kontraktualismus:} Rawls' Form des Kontraktualismus beruht auf der Vorstellung, dass nur solche Normen gerechtfertigt sind, denen alle unter fairen Bedingungen zustimmen würden. Es geht darum, Grundsätze zu identifizieren, auf die sich Menschen einigen könnten, die zwar eigennützig handeln, aber auch Gerechtigkeitssinn besitzen.
	\item \textbf{Konstruktivismus:} Diese Position sieht moralische Normen als Resultat eines bestimmten Rechtfertigungsverfahrens. Die Person als rationaler Akteur ist dabei zentral, das Ergebnis dieses Verfahrens determiniert die Inhalte der obersten Gerechtigkeitsgrundsätze.
	\item \textbf{Urzustand und Schleier des Nichtwissens:} Der hypothetische Urzustand ist ein faires Verfahren, in dem Individuen ohne Kenntnis über ihre persönlichen Umstände (unter dem Schleier des Nichtwissens) Prinzipien ihres Zusammenlebens festlegen. Dies soll zu Grundsätzen führen, die Verteilungsgerechtigkeit gewährleisten.
	\item \textbf{Gesellschaftliche Grundgüter:} Grundgüter sind Mittel, die ein vernünftiger Mensch für die Verfolgung seines Lebensplans als notwendig erachtet; dazu zählen Grundrechte, Freizügigkeit, Einkommen und Besitz sowie soziale Grundlagen von Selbstachtung.
	\item \textbf{Zwei Gerechtigkeitsgrundsätze:} Es handelt sich um ein egalitaristisches Prinzip, in dem der erste Grundsatz allen Menschen gleiche Grundfreiheiten zusichert, während der zweite Grundsatz soziökonomische Ungleichheiten nur zulässt, wenn sie jedem zum Vorteil gereichen (Differenzprinzip).
	\item \textbf{Maximin-Strategie:} Dies ist eine Entscheidungsregel unter Unsicherheit, die besagt, dass die Person den Zustand wählen sollte, der den schlechtestgestellten Individuen die besten Bedingungen bietet.
\end{itemize}

\subsection{Gegenargumente und Probleme}

\begin{itemize}
	\item \textbf{Alternative Gerechtigkeitskonzeptionen:} Gleichheit mag nicht das einzige Prinzip sein, und es könnte Konzepte geben, die dafür plädieren, dass ein menschenwürdiges Leben für alle ausreicht (Nonegalitarismus, Humanismus). Zudem ist fraglich, ob sich alle rationalen Vertragspartner tatsächlich auf Rawls' Gleichheitsprinzip einigen würden.
	\item \textbf{Ausschluss bestimmter Gesellschaftsmitglieder:} Rawls' Theorie schließt möglicherweise Individuen aus, die nicht zur gesellschaftlichen Kooperation fähig sind, wie etwa geistig schwerst Behinderte oder Tiere.
	\item \textbf{Fokus auf Grundgüter statt Fähigkeiten:} Kritiker argumentieren, dass Gleichheit der Fähigkeiten oder die Möglichkeit, Bedürfnisse zu befriedigen, ein relevanteres Ziel sein könnte als die Gleichheit von Gütern.
	\item \textbf{Monologische Verfahren:} Kritiken beinhalten, dass der Urzustand die moralischen Akteure ihrer Identität entfremden könnte und dass das Durchspielen des Urzustandsmodells nicht wirklich eine Transzendierung des eigenen Standpunkts ermöglicht.
	\item \textbf{Kritisches Menschenbild:} Rawls' Annahmen über das Entscheidungsverhalten der Menschen basieren auf der Ökonomischen Theorie der rationalen Wahl, was als einseitig und reduktionistisch angesehen werden könnte.
\end{itemize}

\subsection{Bezug zu Naturethik}

Rawls selbst hat keine ausformulierte Position zur Naturethik und zum moralischen Status von Tieren und der nichtmenschlichen Natur. Seine Theorie könnte jedoch angepasst und erweitert werden, um Fragen der Naturethik und der intergenerationellen Gerechtigkeit einzubeziehen.

\newpage
\section{Aristotelismus klassisch}

\begin{itemize}
	\item \textbf{Teleologische Ethik:}
	      Bei der Aristotelischen Ethik geht es um das Streben nach dem Guten überhaupt und versteht sich als eine Untersuchung, die das Ziel des menschlichen Lebens – die Eudaimonia oder Glückseligkeit – als oberstes Gut definiert. Die Ethik fragt nach dem Sinn und Zweck (Telos) des Lebens, wobei das gute Leben durch das Ausüben der Vernunft und das Vorleben von Tugenden erreicht wird.

	\item \textbf{Eudaimonia (Glückseligkeit):}
	      Eudaimonia bedeutet ein gelingendes, gutes Leben und wird oft mit Glückseligkeit oder Wohlergehen gleichgesetzt. Für Aristoteles ist dies das höchste Ziel menschlichen Strebens, welches um seiner selbst willen erstrebt wird und als selbstgenügsam gilt, also ohne Zufügung weiterer Güter vollkommen ist.

	\item \textbf{Ergon-Argument:}
	      Das Ergon-Argument stützt sich auf die Annahme, dass jedes Wesen oder jede Sache eine spezifische Funktion (Ergon) hat. Für den Menschen ist die Ergon die Ausübung der Vernunft, somit ist die Ausübung der Vernunft bzw. das Leben gemäß der Vernunft das Maß für ein glückliches Leben.

	\item \textbf{Tugenden (Aretê):}
	      Tugenden sind bei Aristoteles sowohl ethische als auch intellektuelle Verhaltensweisen, die ein vernunftgemäßes Handeln repräsentieren. Sie werden durch Gewöhnung und Übung erworben und definieren sich als mittlere Haltung zwischen zwei extremen Reaktionen in gegebenen Situationen.

	\item \textbf{Praktische Klugheit (Phronesis):}
	      Praktische Klugheit ist das vernunftbegabte Überlegen (Prohairesis) hinsichtlich des Handelns in konkreten Situationen und erfordert die Kenntnis um das richtige Maß von Tugenden. Dies ermöglicht eine angemessene Entscheidung und Handlung in Bezug auf ein gutes Leben.

	\item \textbf{Politik und Polis:}
	      Ein glückliches Leben gemäß Aristoteles benötigt ein Leben im Gemeinwesen (Polis), in dem Menschen durch die politische Gemeinschaft ihre Sprach- und Vernunftfähigkeit entfalten können. Die Polis dient somit nicht nur dem Überleben sondern der Möglichkeit, ein gutes Leben im Sinne der Eudaimonia zu führen.
\end{itemize}

\subsection{Probleme der Aristotelischen Konzeption}

\begin{itemize}
	\item Die Hierarchisierung der Lebensformen führt zu einer Bevorzugung der theoretischen Lebensweise, was die inadäquate Betrachtung anderer Lebensweisen wie die der politischen oder Lust-orientierten Lebensform bedeutet.

	\item Der Essentialismus der Aristotelischen Ethik ist kritisch zu betrachten, da Aussagen über das Wesen des Menschen zu einem gegebenen Zeitpunkt schwer zu begründen und historisch wandelbar sind.

	\item Das Fehlen der Einbeziehung leidensfähiger Wesen, insbesondere Tiere, in den Bereich des moralisch Relevanten zeigt eine Grenze der Aristotelischen Ethik auf und trägt zum Anthropozentrismus bei.

	\item Die praktische Anwendbarkeit der Aristotelischen Ethik in Bezug auf gegenwärtige ethische Fragestellungen, insbesondere in der Angewandten Ethik, ist nicht klar definiert und bedarf weiterer Konkretisierung.
\end{itemize}

\subsection{Bezug zur Naturethik}

Während Tiere bei Aristoteles eine sekundäre Rolle spielen und vorwiegend im Kontext ihrer Leidensfähigkeit Beachtung finden, bleibt die unbelebte Natur weitgehend unberücksichtigt. Dies weist auf einen Anthropozentrismus hin, der den Eigenwert der Natur vernachlässigt und primär den Nutzen für den Menschen in den Vordergrund stellt.


\newpage
\section{Aristotelismus modern: Martha Nussbaum}

\begin{itemize}
	\item \textbf{Der „Fähigkeiten-Ansatz“ („capability approach“)}
	      Martha Nussbaum entwickelt eine Ethik und politische Philosophie, die darauf abzielt, das gute menschliche Leben zu verstehen. Sie stützt sich auf eine „dichte vage Theorie des Guten“, die im Gegensatz zu Utilitarismus und Liberalismus steht. Die Theorie orientiert sich an universellen menschlichen Fähigkeiten und deren Realisierung.

	\item \textbf{Aristotelischer Essentialismus}
	      Nussbaum greift auf Aristoteles zurück und nimmt an, menschliches Leben sei durch bestimmte, universale Charakteristiken definiert – die Fähigkeiten oder Vermögen. Ein gutes Leben ist demnach durch die Realisierung dieser Fähigkeiten gekennzeichnet, die im Wesentlichen durch die menschliche Natur bestimmt sind.

	\item \textbf{Liste der Fähigkeiten}
	      Nussbaum formuliert eine Liste elementarer menschlicher Fähigkeiten, die für ein würdiges Leben notwendig sind. Diese Fähigkeiten umfassen unter anderem Lebensverlängerung, Gesundheit, Vernunft, Emotionen, soziale Beziehungen, Spiel und Umgang mit der Natur.

	\item \textbf{Aristotelische Sozialdemokratie}
	      Ein Staat, der Nussbaums Idealen nachkommt, unterstützt die Menschen in der Realisierung ihrer Fähigkeiten. Dieser gerechte Staat wäre ein demokratischer Wohlfahrtsstaat, der einen universellen Anspruch an menschlicher Gerechtigkeit formuliert, was Inklusion und Partizipation beinhaltet.

	\item \textbf{Kritik an klassischer Entwicklungshilfe-Politik}
	      Nussbaum, zusammen mit dem Ökonomen Amartya Sen, kritisiert die Reduktion von Lebensqualität auf das Bruttonationaleinkommen pro Kopf. Sie argumentieren, dass wahre Lebensqualität auch Zugang zu Bildung, Gesundheitsversorgung und politische Partizipation einschließen muss.
\end{itemize}

\subsection{Probleme und Gegenargumente}

\begin{itemize}
	\item \textbf{Kulturrelativistischer Einwand}
	      Es wird bemängelt, dass Nussbaums Liste historische und kulturelle Unterschiede nicht ausreichend berücksichtigt. Nussbaum setzt dagegen, dass ihre Liste kulturelle Spezifikationen zulässt und in einem dialogischen Prozess entstanden ist.

	\item \textbf{Liberalistischer bzw. anti-paternalistischer Einwand}
	      Kritiker befürchten, dass Nussbaums Ansatz Autonomie und Freiheit der Einzelnen untergräbt. Nussbaum weist darauf hin, dass ihr Ansatz die Autonomie schützt und erst die Bereitstellung von Ressourcen wirkliche Freiheit ermöglicht.

	\item \textbf{Anwendung und Missbrauch}
	      Es könnte der Vorwurf erhoben werden, dass die Liste in einer Weise genutzt werden könnte, die bestimmte Gruppen aus der Definition des „Menschlichen“ ausschließt. Diesem Punkt begegnet Nussbaum mit dem Argument, dass eine konkrete Liste die Anerkennung der Menschlichkeit eher fördert.

	\item \textbf{Unterscheidung zwischen wesentlichen und optionalen Fähigkeiten}
	      Einige Kritiker sehen in der Liste eine Vermischung von wesentlichen Lebensaspekten und optionalen Bereichen. Nussbaum könnte entgegenhalten, dass alle Punkte auf der Liste wesentliche Aspekte eines guten menschlichen Lebens darstellen.
\end{itemize}

\subsection{Bezug zur Naturethik}

Im Kontext der Naturethik betont Nussbaum die Bedeutung der menschlichen Beziehung zur Natur. Dies spiegelt sich in ihrer Liste der Fähigkeiten wider, in der die Fähigkeit, in Verbindung zur Natur zu leben, als integraler Bestandteil eines guten menschlichen Lebens aufgeführt wird.



\end{document}
